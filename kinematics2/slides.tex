% the sample slide is created with 16:9 aspect ratio
\documentclass[aspectratio=169]{beamer}

\usepackage{mathpazo}
\usepackage{amssymb}
\usepackage{mathtools}
\usepackage[makeroom]{cancel}

\usetheme{saarland}

% the background and logo are in the images directory
\graphicspath{{figs/}}

% information for the title page
\author{Jason Yu}
\title{AP FRQ}
\subtitle{Worked problem 15}
\institute{University HS}
\date{\today}

\begin{document}
	\begin{frame}{Problem 15}
		A system consists of a ball of mass \(M_2\) and a uniform rod of mass \(M_1\) and length \(d\). The rod is attached to a horizontal frictionless table by a pivot at point \(P\) and initially rotates at an angular speed \(\omega\), as shown. The rotational inertia of the rod about point \(P\) is \(\frac{1}{3} M_1 d^2\). The rod strikes the ball, which is initially at rest. As a result of this collision, the rod is stopped and the ball moves in the direction shown. Express all answers in terms of \(M_1\), \(M_2\), \(\omega\), \(d\), and fundamental constants.
		\begin{enumerate}[a)]
			\item Derive an expression for the angular momentum of the rod about point \(P\) before the collision.
			\item Derive an expression for the speed \(v\) of the ball after the collision.
			\item Assuming that this collision is elastic, calculate the numerical value of the ratio \(M_1/M_2\).
			\item A new ball with the same mass \(M_1\) as the rod is now placed a distance \(x\) from the pivot, as shown. Again assuming the collision is elastic, for what value of \(x\) will the rod stop moving after hitting the ball?
		\end{enumerate}
	\end{frame}

	\begin{frame}{Problem 15}
		\begin{enumerate}[a)]
			\item Derive an expression for the angular momentum of the rod about point \(P\) before the collision.
		\end{enumerate}

		\begin{align*}
			\onslide<2->{L &= I \omega \\}
			\onslide<3->{\Aboxed{L &= \left(\frac{1}{3}M_1 d^2\right)(\omega)}}
		\end{align*}
		
	\end{frame}

	\begin{frame}{Problem 15}
		\begin{enumerate}[a)]
			\setcounter{enumi}{1}
			\item Derive an expression for the speed \(v\) of the ball after the collision.
		\end{enumerate}

		\begin{align*}
			\onslide<2->{L_\text{rod} &= L_\text{ball} \\}
			\onslide<3->{\left(\frac{1}{3}M_1d^2\right)(\omega) &= M_2 v d \\}
			\onslide<4->{\Aboxed{v &= \frac{M_1d\omega}{3M_2}}}
		\end{align*}
	\end{frame}

	\begin{frame}{Problem 15}
		\begin{enumerate}[a)]
			\setcounter{enumi}{2}
			\item Assuming that this collision is elastic, calculate the numerical value of the ratio \(M_1/M_2\).
		\end{enumerate}
		
		\begin{align*}
			\onslide<2->{K_\text{rod} &= K_\text{ball} \\}
			\onslide<3->{\frac{1}{2}M_1v_\text{rod}^2 &= \frac{1}{2}M_2v_\text{ball}^2 \\}
			\onslide<4->{\frac{1}{2}M_1(\omega d)^2 &= \frac{1}{2}M_2 v_\text{ball}^2 \\}
			\onslide<5->{1 &= \frac{M_1}{3M_2} \\}
			\onslide<6->{\Aboxed{\frac{M_1}{M_2} &= 3}}
		\end{align*}
	\end{frame}

	\begin{frame}{Problem 15}
		\begin{enumerate}[a)]
			\setcounter{enumi}{3}
			\item A new ball with the same mass \(M_1\) as the rod is now placed a distance \(x\) from the pivot, as shown. Again assuming the collision is elastic, for what value of \(x\) will the rod stop moving after hitting the ball?
		\end{enumerate}

		\begin{align*}
			\onslide<2->{L_\text{rod} &= L_\text{ball} \\}
			\onslide<3->{\frac{1}{3}Md^2 \omega &= Mvx \\}
			\onslide<4->{\Aboxed{v &= \frac{d^2 \omega}{3x}}}
		\end{align*}
	\end{frame}

	\begin{frame}{Problem 15}
		\begin{equation*}
			v = \frac{d^2 \omega}{3x}
		\end{equation*}
		\begin{align*}
			\onslide<2->{E_\text{i} &= E_\text{f} \\}
			\onslide<3->{\frac{1}{2} I \omega^2 &= \frac{1}{2} Mv^2 \\}
			\onslide<4->{\left(\frac{1}{3}Md^2\right) \omega^2 &= Mv^2 \\}
			\onslide<5->{\sqrt{\frac{1}{3}} d\omega &= v \\}
			\onslide<6->{\sqrt{\frac{1}{3}} d\omega &= \frac{d^2\omega}{3x} \\} 
			\onslide<7->{\Aboxed{x &= \frac{d}{\sqrt{3}}}}
		\end{align*}
	\end{frame}
\end{document}
