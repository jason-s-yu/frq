\documentclass[aspectratio=169]{beamer}

\usepackage{mathpazo}
\usepackage{amssymb}
\usepackage{mathtools}
\usepackage[makeroom]{cancel}
\usepackage{siunitx}

\usetheme{saarland}

\graphicspath{{figs/}}

\author{Jason Yu}
\title{AP Magnetism Review}
\subtitle{Worked problem 9}
\institute{University HS}
\date{\today}

\begin{document}
	\begin{frame}{Problem 9}
		Your research director has assigned you to set up the laboratory's mass spectrometer so that it will separate strontium ions having a net charge of \(+2e\) from a beam of mixed ions. The spectrometer above accelerates a beam of ions from rest through a potential difference \(\epsilon\), after which the beam enters a region containing a uniform magnetic field \(\vec{B}\) of constant magnitude and perpendicular to the plane of the path of the ions. The ions leave the spectrometer at a distance \(x\) from the entrance point. You can manually change \(\epsilon\).

		\begin{enumerate}[a)]
			\item In what direction must \(\vec{B}\) point to produce the trajectory of the ions shown?
			\item The ions travel at constant speed around the semicircular path. Explain why the speed remains constant.
			\item Calculate the speed of the ions with charge \(+2e\) that exist at distance \(x\).
			\item Calculate the accelerating voltage \(\epsilon\) needed for the ions with charge \(+2e\) to attain the speed you calculated in part c.
		\end{enumerate}
	\end{frame}

	\begin{frame}{Problem 9}
		\begin{enumerate}[a)]
			\item In what direction must \(\vec{B}\) point to produce the trajectory of the ions shown?
		\end{enumerate}

		\begin{itemize}
			\item The particle goes towards the left, so a force must be acting upon it towards the left
			\item Thumb towards left and index finger directs upward (in direction of particle's velocity)
		\end{itemize}

		\begin{equation*}
			\onslide<2->\boxed{\text{Into the page}}
		\end{equation*}
	\end{frame}

	\begin{frame}{Problem 9}
		\begin{enumerate}[a)]
			\setcounter{enumi}{1}
			\item The ions travel at constant speed around the semicircular path. Explain why the speed remains constant.
		\end{enumerate}

		\begin{equation*}
			\onslide<2->\boxed{\text{Force is perpendicular to } \vec{v} \text{ so it does no work}}
		\end{equation*}
	\end{frame}

	\begin{frame}{Problem 9}
		\begin{enumerate}[a)]
			\setcounter{enumi}{2}
			\item Calculate the speed of the ions with charge \(+2e\) that exist at distance \(x\).
		\end{enumerate}

		\begin{align*}
			\onslide<2->{\frac{mv^2}{R} &= qvB \\}
			\onslide<3->{\frac{mv}{x/2} &= qB \\}
			\onslide<4->{v &= \frac{qBx}{2m} \\}
			\onslide<4->{v &= \frac{(2e)(\SI{0.070}{\tesla})(\SI{1.35}{\meter})}{2(\SI{1.45E-25}{\kilo\gram})} \\}
			\onslide<5->{\Aboxed{v &= \SI{1.04E5}{\meter/\second}}}
		\end{align*}
	\end{frame}

	\begin{frame}{Problem 9}
		\begin{enumerate}[a)]
			\setcounter{enumi}{3}
			\item Calculate the accelerating voltage \(\epsilon\) needed for the ions with charge \(+2e\) to attain the speed you calculated in part c.
		\end{enumerate}

		\begin{align*}
			\onslide<2->{q\epsilon &= \frac{1}{2}mv^2 \\}
			\onslide<3->{\epsilon &= \frac{mv^2}{2q} \\}
			\onslide<4->{\epsilon &= \frac{(\SI{1.45E-25}{\kilo\gram})(\SI{1.04E5}{\meter/\second})^2}{4e} \\}
			\onslide<5->{\Aboxed{\epsilon &= \SI{2400}{\volt}}}
		\end{align*}
	\end{frame}
\end{document}
